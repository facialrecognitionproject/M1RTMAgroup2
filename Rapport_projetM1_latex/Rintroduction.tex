\chapter*{Avant-propos}
%\addcontentsline{toc}{chapter}{Introduction}
%\minitoc

At the University of Poitiers, students in Master of Information and Communication Systems, Emphasis  (professional and research career) in NetworkTechnologies field have to carry out a project during their first year in order to complete the training.This is the mean reason we have being written this report of project.   
\paragraph{}
In fact, several topics were proposed to us and we choose the facial recognition for its multiplicity and variety in term of application fields, such as high security applications, remote monitoring and control of access etc. Indeed, this field is closely linked with human vision reproduction researche and the interaction between a machine and a human, which is a fascinating and booming subject.
\paragraph{}
Facial recognition consist in identifying a person with a picture of his face. It appears to be a quite active field in Vision from computer and Biomety. Recognition with an individual face is a skill of Biometry which is still to improve. Indeed, the acquired (captured) picture represents variation much higher than other characteristics (makeup, presence or absence of glasses, aging and expression of emotion). The method of face recognition is sensitive to the variation of the light and change the position of the face in the image acquisition. But still the system obtained is not yet able to adapt itself to certain kinds of variations on a face.

\section{Goals or objectives :}
Several methods have been developed in the literature for face recognition. In our work, we chose two extraction techniques characteristics of the face image:
The first method is Eigenface which is based on principal component analysis. The CPA is a mathematical method which can be used to simplify a dataset, reducing its size.
\paragraph{}
The second method is Fisherface which is based on a linear discriminant analysis. LDA is improved PCR method.
\paragraph{}
These two methods should therefore allow us to achieve two prototypes we compare the results. This step is to assess which method provides a more satisfactory result in terms of facial recognition.
\section{Work methodologies :}
To achieve the two prototypes necessary for our analysis we have adopted the following plan personable this report. First, we define the project environment and aims at first, then we develop the state of the art methods to be programmed, as it were the theoretical concept. Then we establish a technical report is to explain the use of our prototypes made and finally we leave the analysis of the use of these prototypes to the comparison of our results.
