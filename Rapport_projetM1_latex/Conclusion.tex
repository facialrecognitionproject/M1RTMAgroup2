\chapter*{Conclusion}
%\addcontentsline{toc}{chapter}{Conclusion}
%\minitoc

The objectif of the project was to develop a facial capture software program (face recognition) in XLIM-SIC laboratory research work in collaboration with a company based in Lyon operating in the field of manufacturing tablets for children.
\paragraph{}
To achieve this project, we used two methods to implement: the eigenfaces and Fisherfaces.
The eigenfaces using principal component analysis are a set of eigenvectors used in the field of vision to solve the problem of face recognition. The eigenfaces allow  to reconstruct a subspace retaining the best eigenvalues while keeping much useful information.
The Fisherfaces in turn use the LDA that analyze eigenvectors of the scatter matrix and aims to maximize the inter-class variations while minimizing the intra-class variations.
\paragraph{}
We have started with the implementation of  eigenfaces method in python. For that it was necessary for us to  have an image database. we used the ATT image database. However, some difficulties were encountered during programming which delayed us on objectives. That  why we focused on eigenfaces method and we could not make a comparison of the results of both methods


%\paragraph{}


