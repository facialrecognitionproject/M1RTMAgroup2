\chapter{Presentation of the project}
%\minitoc


\section{Context and scope of work}
\subsection{Context}
Previously carried out at the SIC Department of XLIM laboratory, the project we are in charge of implements "algorithmic tools and capturing software for facial recognition". This project interest was also interactions with and between users of a video game with an educational aim ("Serious game") by automatically animated avatars and experienced analysis difficulties, as well as the development of a software library for developers of video games on smartphones.
\paragraph{}
Facial recognition is often used as a tool to secure devices or control the use of applications. This technique both helps adults secure their systems and devices and parents limit and control access to systems for their kids.
The main idea was to improve parental control of tablets for children’s use. This could be helpful for parents wishing to use facial recognition to limit their children’s use past a certain time of a day.
\subsection{Scope of work}
This project aims to develop a facial capture software program (face recognition) in XLIM-SIC laboratory research work in collaboration with a company based in Lyon operating in the field of manufacturing tablets for children .
The required results at the end of this project are: facial recognition software program with
\begin{itemize}
\item Eigen Faces;
\item Fisher Faces.
\end{itemize}
Those will be programmed with multiple programming languages (Python, C ++).
This project is performed with the research professors of the Fundamental Faculty of  Sciences at the University of Poitiers: Pascal Bourdon and David HELBERT who  supervises the project


\clearpage


\section{Objectives }
The work required for this project is the development of a share recording software for the recognition of facial expressions for the identification of a face from an ID list in a home or for parental control.
Today innovation and the latest technologies are growing increasingly and allow the interactivity between human-machine to be maximal. Research  on facial expression is fundamental in many applications.
Facial recognition takes place in three stages, namely:
\begin{itemize}
\item Face detection
\item Extraction and normalization of facial features
\item Identification and / or verification
\end{itemize}
\paragraph{}
The main difficulty in face recognition is that  there are no two identical faces. Thus, each individual is unique and will be marked by gender, ethnicity, age or his haircut, but also by the shape, size and arrangement of the elements of the face.
\paragraph{}
This project has an educational goal because it contributes greatly to our engineering multimedia training, allowing us to put into practice the theories studied in the various teaching modules of our maste’sr degree (Image processing, tool and scientific computation, Algorithms for multimedia, random signal processing) but also by completing them. This project can be seen as a complement to our training.



\section{Deadlines}

 \begin{table}[!ht]%
\begin{center}
\begin{tabular}{|c|c|}
  \hline
  Deadlines & Deliverables  \\
  \hline
  \scriptsize{April 2} & Users requirements presentation + handouts \\
   \hline
  \scriptsize{May 13} & State of the art Theoretical \\
   \hline
  \scriptsize{June 11} & Official delivery (technical documents, codes, ...)\\
   \hline
  \scriptsize{June 18} & Project presentation\\
   \hline
   
\end{tabular}
\end{center}
\caption{\textbf{Deadline Table}}
\label{tab1}
\end{table}


